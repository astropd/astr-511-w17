\documentstyle[12pt,aaspp4]{article}

%%%%%%%%%%%%%%%%%%%%%%%%%%%%%%%%%%%%%%%%%%%%%%%%%%%%
%%% author-defined commands
\newcommand\about     {\hbox{$\sim$}}
\newcommand\x         {\hbox{$\times$}}
\newcommand\othername {\hbox{$\dots$}}
\def\eq#1{\begin{equation} #1 \end{equation}}
\def\eqarray#1{\begin{eqnarray} #1 \end{eqnarray}}
\def\eqarraylet#1{\begin{mathletters}\begin{eqnarray} #1 %
                  \end{eqnarray}\end{mathletters}}
\def\non    {\nonumber \\}
\def\DS     {\displaystyle}
\def\E#1{\hbox{$10^{#1}$}}
\def\sub#1{_{\rm #1}}
\def\case#1/#2{\hbox{$\frac{#1}{#2}$}}
\def\about  {\hbox{$\sim$}}
\def\x      {\hbox{$\times$}}
\def\ug               {\hbox{$u-g$}}
\def\gr               {\hbox{$g-r$}}
\def\ri               {\hbox{$r-i$}}
\def\iz               {\hbox{$i-z$}}
\def\a                {\hbox{$a^*$}}
\def\O                {\hbox{$O$}}
\def\E                {\hbox{$E$}}
\def\Oa               {\hbox{$O_a$}}
\def\Ea               {\hbox{$E_a$}}
\def\Jg               {\hbox{$J_g$}}
\def\Fg               {\hbox{$F_g$}}
\def\J                {\hbox{$J$}}
\def\F                {\hbox{$F$}}
\def\N                {\hbox{$N$}}
\def\dd               {\hbox{deg/day}}
\def\mic              {\hbox{$\mu{\rm m}$}}
\def\Mo{\hbox{$M_{\odot}$}}
\def\Lo{\hbox{$L_{\odot}$}}
\def\comm#1           {\tt #1}
\def\refto#1          {\ref #1}
\def\T#1              {({\bf #1})}
\def\H#1              {({\it #1})}

%%%%%%%%%%%%%%%%%%%%%%%%%%%%%%%%%%%%%%%%%%%%%%%%%%%%

\begin{document}

\title{ {\bf Term Project \# 1}}  
                   
\centerline{Galactic Astronomy (Astr 511); Winter Quarter 2017}
\centerline{prof. Mario Juri\'{c} and prof. \v{Z}eljko Ivezi\'{c}, University of Washington} 

The text data file Astr511HW1data.dat (linked to class webpage as
a gzipped file) contains SDSS measurements for 
about 600,000 stars with $b>80^\circ$ (i.e. within 10$^\circ$ 
from the north galactic pole) and $14<r<21$. The data are listed as one line
per star, with each line containing the following quantities:
\begin{itemize}
\item {\bf ra dec:} right ascension and declination (J2000.0)
 in decimal degrees 
\item {\bf run:} SDSS observing night identifier
\item {\bf Ar:} the value of the $r$ band ISM extinction used to 
     correct photometry (adopted from the SFD maps; for bands 
     other than $r$ standard SDSS coefficients are used)
\item {\bf u g r i z:} SDSS photometry (corrected for the ISM
       extinction)
\item {\bf uErr gErr rErr iErr zErr:} photometric errors
\item {\bf pmL pmB:} proper motion vector components in the
      longitudinal and latitudinal directions (mas/yr); set to
      999.99 when no measurement is available 
\item {\bf pmErr:} mean proper motion error (mas/yr); set to
      999.99 when no measurement is available 
\end{itemize}

For stars from this file, compute absolute magnitude using
a photometric parallax relation, $M_r(g-i,[Fe/H])$, given
by eqs. A2, A3 and A7 from Ivezi\'{c} et al. 2008 (ApJ, 684, 287).
For computing metallicity, $[Fe/H]$, instead of their
eq.~4, use an updated expression from Bond et al. 2010 (ApJ, 716, 1): 
\begin{equation}
\label{Zphotom}
  [Fe/H] = A + Bx + Cy + Dxy  + Ex^2 + Fy^2 + Gx^2 y + Hxy^2 + Ix^3 + Jy^3,
\end{equation}
with $x=(u-g)$ and $y=(g-r)$, and the best-fit coefficients ($A$--$J$) = 
($-$13.13, 14.09, 28.04, $-$5.51, $-$5.90, $-$58.68, 9.14, $-$20.61, 0.0,
58.20). This expression if valid only for $g-r<0.6$; for redder stars
use $[Fe/H]=-0.6$. 

Since $b>80^\circ$, for these stars the distance from the galactic plane, $Z$,
and the distance from us, $D$, are approximately the same. Using $Z=D$, where
$D$ is computed from $r-M_r= 5*\log{\left(D/({\rm 10 pc})\right)}$, do the following:
\begin{enumerate}
\item For stars with $0.2<g-r<0.4$, plot $\ln(\rho)$ vs. $Z$, where 
  $\rho$ is the stellar number density in a given bin (e.g. look at 
  Figs. 5 and 15 in Juri\'{c} et al. 2008, ApJ, 673, 864 for similar
  examples). You can approximate
  $\rho(Z) = N(Z)/V(Z)$, where $N(Z)$ is the number of stars in a given
  bin, and $V(Z)$ is the bin volume (note that the solid angle is 
  $\Delta \Omega \sim 314$ deg$^\circ$). What is the $Z$ range where you
  believe the results, and why?
\item Add $\ln(\rho)$ vs. $Z$ for stars with $0.4<g-r<0.6$, $0.6<g-r<0.8$,
      and $0.8<g-r<1.0$ (you can rescale all curves to the same value
      at some fiducial $Z$, or leave them as they are). Discuss the 
      differences compared to the $0.2<g-r<0.4$ subsample. Why do we
      expect larger systematic errors for $0.8<g-r<1.0$ than for the
      adjacent bin with $0.4<g-r<0.6$?
\item For subsample with $0.2<g-r<0.4$, separate stars into low-metallicity
      sample, $[Fe/H]<-1.0$, and high-metallicity sample, $[Fe/H]>-1.0$. 
      Compare their $\ln(\rho)$ vs. $Z$ curves. What do you conclude?
\item For these low-metallicity and high-metallicity samples, plot and 
      compare their differential $r$ band magnitude distributions (i.e. the 
      number of sources per unit magnitude, in small, say 0.1 mag wide, $r$
      bins). What do you conclude? How would you numerically describe 
      these curves (i.e. what kind of functional form for the fitting
      functions would you choose)? 
\item What should be the faint $r$ band limit for a survey to be able to
      map the $\ln(\rho)$ vs. $Z$ profile out to 100 kpc using main-sequence
      stars? Assume the same color distribution as for the SDSS sample. 
      For a solid angle of 1 deg$^2$, how many stars with $0.2<g-r<0.4$
      would you expect with distances between 90 kpc and 100 kpc? 
      Assume whatever additional information you need to solve this 
      problem (not all required information is provided here).
\end{enumerate}



\end{document}


From Reid \& Hawley:

  v_x =   v_l sin(l)  +  v_b sin(b) cos(l) - v_r cos(b) cos(l)

  v_y = - v_l cos(l)  +  v_b sin(b) sin(l) - v_r cos(b) sin(l)

  v_z =   v_b cos(b)  +  v_r sin(b) 

 ansatz: if v_z=0, v_r = - v_b cos(b) / sin(b)





    HW:



IDL or other code, reading a number of vectors with several
million elements from files; simple operations with vectors
such as binning and low-order statistics; visualization


1  density law from counts 
   prepare SDSS file with ra, dec, ugriz for 0.2<g-r<0.6 and b>70
   compute FeH and distance using tomography (and appendix)
   plot ln(rho) vs. Z for low and high FeH, separately for 
   g-r=0.2-0.4 and 0.4-0.6, compare 
   
   take LSST simulated sample, how well can we measure halo profile?

2  WD luminosity function
   take LSST simulation, fit photometric parallax, apply to deeper
   sample, separate by vtan into halo and disk, compute LFs (Cminus)

3  S82 metallicity distributions 
   for coadded file with 0.2<g-r<0.4 compute FeH and 
   plot median, rms, fraction of >+-2 sigma outliers
   in the polar plot (RA vs D), over plot lines of 
   constant l and b

4  Potentials from Jeans equations 
   For 0.2<g-r<0.4 sample in the meridional plane, 
   compute velocity dispersions, plug into Jeans equations,
   get potential gradient separately for D and H stars


Astr 509:
1: counts, p5-11 
2: v_c etc for spherical systems, p13-23 (also p. 9-15 in lec 3)
4: refs for MW potential, p2
5 and 6: orbits of stars!
7 and 8 and 9: derivation of Boltzman eq and Jeans eqs.
10: virial theorem
11: King profiles and Jeans instability
12: spiral arms
14: galaxy collisions
15: scattering of disk stars

Astr 598:
1: measurement of I, p2-4, add calib doc
2: types of mags, p11-16
5: stars, separating pops
7: halo structure
8: galaxies in SDSS
12: clusters of galaxies

Astr323:
2: intro to MW, and ISM
4: spiral gals, 5: ellipticals
6: gals in SDSS and LF
8: gal formation and interactions


  \item {\bf Tue: Jan 13}  Introduction, review of stellar astrophysics
     photom: 598/1,2; sep pop: 598/5, virial theorem 509/10
  \item {\bf Thu: Jan 15}  Review of galaxies
     S and E: 323;4,5 spiral arms in 509/12, 323/8, clusters 598/12
  \item {\bf Tue: Jan 20}  Galaxies in SDSS, luminosity function
     598/8, 323/6 
  \item {\bf Thu: Jan 22}  Basic properties of the Milky Way (MW)
    intro 323/2, counts 509/1, MW potential, halo struct 598/7
  \item {\bf Tue: Jan 27}  Basic properties of MW ISM (including SFD maps) {\bf HW 1 due}
     323/2  add extinction properties Mie theory 
  \item {\bf Thu: Jan 29}  Globular clusters and other simple stellar pops
      509/11 models from Bruzual, Girardi (papers?) 


  \item {\bf Tue: Feb  3}  Stellar count distribution in MW: I 
  \item {\bf Thu: Feb  5}  Stellar count distribution in MW: II {\bf HW 2 due}
  \item {\bf Tue: Feb 10}  Guest lecturer: ???
  \item {\bf Thu: Feb 12}  Stellar metallicity distribution in MW
  \item {\bf Tue: Feb 17}  Stellar kinematics in MW
  \item {\bf Thu: Feb 19}  Merger history: observations and theory {\bf HW 3 due}
  \item {\bf Tue: Feb 24}  Jeans Equations and Applications
  \item {\bf Thu: Feb 26}  Stellar Orbits, Scattering of Disk Stars
  \item {\bf Tue: Mar  3}  Guest lecturer: Numerical approach to galaxy formation
  \item {\bf Thu: Mar  5}  The Road Ahead: Gaia and LSST {\bf HW 4 due}

















