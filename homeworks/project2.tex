\documentstyle[12pt,aaspp4]{article}

%%%%%%%%%%%%%%%%%%%%%%%%%%%%%%%%%%%%%%%%%%%%%%%%%%%%
%%% author-defined commands
\newcommand\about     {\hbox{$\sim$}}
\newcommand\x         {\hbox{$\times$}}
\newcommand\othername {\hbox{$\dots$}}
\def\eq#1{\begin{equation} #1 \end{equation}}
\def\eqarray#1{\begin{eqnarray} #1 \end{eqnarray}}
\def\eqarraylet#1{\begin{mathletters}\begin{eqnarray} #1 %
                  \end{eqnarray}\end{mathletters}}
\def\non    {\nonumber \\}
\def\DS     {\displaystyle}
\def\E#1{\hbox{$10^{#1}$}}
\def\sub#1{_{\rm #1}}
\def\case#1/#2{\hbox{$\frac{#1}{#2}$}}
\def\about  {\hbox{$\sim$}}
\def\x      {\hbox{$\times$}}
\def\ug               {\hbox{$u-g$}}
\def\gr               {\hbox{$g-r$}}
\def\ri               {\hbox{$r-i$}}
\def\iz               {\hbox{$i-z$}}
\def\a                {\hbox{$a^*$}}
\def\O                {\hbox{$O$}}
\def\E                {\hbox{$E$}}
\def\Oa               {\hbox{$O_a$}}
\def\Ea               {\hbox{$E_a$}}
\def\Jg               {\hbox{$J_g$}}
\def\Fg               {\hbox{$F_g$}}
\def\J                {\hbox{$J$}}
\def\F                {\hbox{$F$}}
\def\N                {\hbox{$N$}}
\def\dd               {\hbox{deg/day}}
\def\mic              {\hbox{$\mu{\rm m}$}}
\def\Mo{\hbox{$M_{\odot}$}}
\def\Lo{\hbox{$L_{\odot}$}}
\def\comm#1           {\tt #1}
\def\refto#1          {\ref #1}
\def\T#1              {({\bf #1})}
\def\H#1              {({\it #1})}

%%%%%%%%%%%%%%%%%%%%%%%%%%%%%%%%%%%%%%%%%%%%%%%%%%%%

\begin{document}


\title{ {\bf Term Project \# 2}}  
                   
\centerline{Galactic Astronomy (Astr 511); Winter Quarter 2017}
\centerline{prof. Mario Juri\'{c} and prof. \v{Z}eljko Ivezi\'{c}, University of Washington} 

This homework is based on two data files that contain information about a simulated sample of 
white dwarfs which mimics LSST observations. The sample generation, including various underlying 
assumptions, is described in the linked document (WDdraft.pdf). These data files (both are compressed 
using gzip utility) contain input information (LSSTsimWDtruth60.dat, 33 MB compressed) and “observed” 
properties (LSSTsimWDobs60.dat, 44 MB compressed). Both data files have identical number of data lines, 
and map onto each other (i.e., the n-th data lines in both files correspond to the same simulated star). 
The sample is defined by $r < 27.5$ and $b > 60^\circ$, and includes 785,760 stars. 

The “truth” data file (LSSTsimWDtruth60.dat) lists the following quantities:
\begin{itemize}
\item {\bf ra dec:} right ascension and declination (J2000.0)
 in decimal degrees 
\item {\bf u g r i z y:} ``true'' magnitudes in LSST bandpasses (based on the
Bergeron white dwarf models, no correction for the ISM extinction)
\item {\bf $M_r$:} absolute magnitude, $M_r$, in the $r$ band (drawn from the 
    Harris et al. luminosity function)
\item {\bf log(g):} set to 8.0 for all stars
\item {\bf $v_R$, $v_\phi$, $v_Z$:} model velocity in galactocentric cylindrical coordinates 
($R$ points away from the galactic center, $Z$ points towards the North Galactic Pole, and 
the coordinate system is right-handed; the Sun is at ($R$=8 kpc, $Z$=25 pc) and the local 
standard of rest rotates with $v_\phi$ = $−$220 km/s). The velocity distribution is drawn 
from the Bond et al. 2010 (ApJ, 716, 1) model.
\item {\bf T:} WD model type; 1=Hydrogen WD, 2=He WD (color tracks depend on this type), 
10\% of the population is randomly assigned T=2.
\item {\bf P:} Galactic population: 1=disk, 2=halo; the population assignment and overall spatial 
distribution is drawn from the Juri\'{c} et al. 2008 (ApJ, 684, 287) model.
\end{itemize}

The “observational” data file (LSSTsimWDobs60.dat) lists the following quantities:
\begin{itemize}
\item {\bf ra dec:} right ascension and declination (J2000.0)  in decimal degrees 
\item {\bf mObs, mErr; m=(u, g, r, i, z, y):} ``observed'' magnitudes, generated by convolving 
``true'' magnitudes with expected LSST errors (not corrected for the ISM extinction). The expected 
errors are computed as described in the LSST overview paper\footnote{https://arxiv.org/abs/0805.2366} 
(Ivezi\'{c} et al. 2008; arXiv:0805.2366).
\item {\bf piObs, piErr:} trigonometric parallax and its expected error, in milliarcsec (the listed 
parallax is generated by convolving the true parallax with expected error; the true parallax is 
computed from true distance, with the latter determined from $M_r$ and true $r$). The parallax 
error is computed as described in the LSST overview paper (see section 3.3.3).
\item {\bf muRAObs, muDecObs, muErr}: the components of the proper motion vector in the 
R.A. and Dec directions, and the proper motion error (per coordinate), in milliarcsec/yr. The proper 
motion is generated using velocity from the first file and by convolving the true proper motion 
with the expected proper motion error (the latter is computed as described in the LSST 
overview paper, see section 3.3.3).
\end{itemize}


Using data from these two files, do the following:

{\bf A)} Define a ``gold parallax sample'' by requiring a signal-to-noise ratio of at least 10 for the 
trigonometric parallax measurement (i.e., $piObs/piErr>10$). Compute the distance and distance 
modulus from the parallax measurement ($D$/kpc=1 milliarcsec/$piObs$) and compare it to the 
distance modulus determined from $r$ and $M_r$ listed in the ``truth'' file. Plot the distribution of 
the distance modulus difference and compute its median and root-mean-square scatter (hint: beware 
of outliers and clip at 3$\sigma$!). Are they ``interestingly'' small? Is the distribution deviating from 
a gaussian? Would you expect it to? Why? How many white dwarfs would you expect in a ``gold parallax 
sample'' from the full LSST survey area of 20,000 deg$^2$ (hint: simply scale by the area because the 
distance cutoff is smaller than the thin disk scaleheight)? Plot the $(g-r)$ vs. $(u-g)$ color-color 
diagram (using observed photometry) for this sample. Does it look crisper than the SDSS distribution 
shown in the bottom left corner of fig.~23 in Ivezi\'{c} et al. (2007, AJ, 134, 973)? Hint: look at the two 
bottom panels in fig. 24.

{\bf B)} Using the ``gold parallax sample'' from {\bf A}, estimate the absolute $r$ band magnitude as 
$Mobs = rObs-DMobs$, with the observed distance modulus, $DMobs$, determined using
the ``measured'' trigonometric parallax, $piObs$. Plot $Mobs$ vs. $(gObs-rObs)$ color for 
stars with $T$=1 (i.e., hydrogen white WDs; while this is a shortcut based on model input, it is 
possible to photometrically distinguish hydrogen from helium WDs by considering their four-dimensional 
color loci; however, this is beyond the scope of this project and hence this shortcut). Fit a low-order 
polynomial to derive a photometric parallax relation, $M_r(g-r)$ (hint: you may want to first compute 
the median $M_r$ in about 0.1 mag wide bins of the $g-r$ color, and then fit a polynomial to these 
median values vs. $g-r$ bin value). How did you choose the order of your polynomial fit? In what 
range of $M_r$ and $(g-r)$ is your relation valid?

{\bf C)} Define a ``gold proper motion sample'' by requiring $rObs < 24.5$. What fraction of
this sample has the observed proper motion measured with a signal-to-noise ratio (to compute
SNR: add the two proper motion components in quadrature and divide by the listed proper
motion error) of at least 3? Apply your photometric parallax relation from {\bf B} to estimate
$M_r$ and distance (using $Mr$ and $rObs$). Use this distance to compute tangential velocity,
$v_{tan}$ (of course, you also need the observed proper motion; be careful about units!). Define
a candidate disk sample as stars with $v_{tan}< v_{tan}^{cutoff}$, and a candidate halo sample as stars with
$v_{tan} > v_{tan}^{cutoff}$. Using $P$ from the “truth” file, plot the completeness and contamination 
for disk and halo samples as a function of $v_{tan}^{cutoff}$ for $0 < v_{tan}^{cutoff} < 500$ km/s (in steps 
of, say, 20 km/s). The completeness is defined as the number of (disk, halo) objects in the selected 
subsample divided by the total number of such objects, and contamination is the number of objects
 of the ``wrong'' type in the selected subsample divided by the total number in that subsample.

{\bf D)} Using the ``gold proper motion sample'' from {\bf C}, define a candidate disk sample by
$v_{tan} < 150$ km/s, and a candidate halo sample by $v_{tan} > 200$ km/s. Using your results 
from {\bf C}, estimate the completeness and contamination for each subsample. Using the 
$C^−$ method implemented in astroML\footnote{See
http://www.astroml.org/book\_figures/chapter4/fig\_lyndenbell\_toy.html} (of course, you
are welcome to write your own code for extra credit and a feeling of an extraordinary accomplishment!), 
compute the differential luminosity function for each subsample (this is the hardest 
part of this project!). Explain how did you get the normalization constant. Plot your results in a
log($\Phi$) vs. $M_r$ diagram (with error bars!), and overplot the “true” luminosity function listed in 
files WDlumfuncDisk.dat and WDlumfuncHalo.dat (the differential LF listed in the second column is 
expressed as the number of stars per pc$^3$ and mag; the LFs are slightly inconsistent with the Harris 
et al. due to a bug in simulations but, importantly, they do correspond to the ``true'' LFs for the simulated 
sample). Comment on (dis)agreement between your $\Phi$ and the true $\Phi$ (which was used to generate 
the simulated sample). {\bf IMPORTANT - YOU NEED TO FIX A BUG:} Divide the halo $\Phi$ from file WDlumfuncHalo.dat 
by 200 to get a proper normalization in units of stars per pc$^3$ and mag! 

{\bf E)} A ``byproduct'' of the luminosity function determination in {\bf D} is the spatial distribution of stars. 
Plot the results for disk and halo subsamples (i.e., ln($\rho$) vs. $Z$, with error bars!). Compare these profiles
 to the spatial profiles you determined in project \#1 and comment.



\end{document}
